\documentclass[12pt]{article} % font size


\usepackage{graphicx} % to include images
\usepackage{float} % to float figures
\usepackage{booktabs,makecell} % for diagonal cells
\usepackage{hyperref} % for hyperlinks
\usepackage{listings} % for including files
\usepackage[top=1in, bottom=1in, left=1.25in, right=1.25in]{geometry} % set margins
\usepackage{amssymb}
\usepackage{amsmath}
\usepackage{mathtools}
\renewcommand{\familydefault}{\sfdefault} % use sans serif per default
\usepackage{helvet} % use helvetica per default

\usepackage[utf8]{inputenc} % for unicode input characters
\usepackage[T1]{fontenc}

\lstset{%
  basicstyle=\scriptsize\sffamily,%
  commentstyle=\footnotesize\ttfamily,%
  frameround=trBL,
  frame=single,
  breaklines=true,
  showstringspaces=false,
  numbers=left,
  numberstyle=\tiny,
  numbersep=10pt,
  keywordstyle=\bf
}


\makeatletter

\makeatother
\usepackage[english]{babel}
\usepackage{graphicx}

\usepackage{booktabs,makecell} % for diagonal cells
\usepackage{hyperref} % for hyperlinks

\renewcommand{\familydefault}{\sfdefault} % use sans serif per default
\usepackage{helvet} % use helvetica per default

\usepackage[utf8]{inputenc} % for unicode input characters
\usepackage[T1]{fontenc}

\makeatletter

\makeatother

\begin{document}

\author{SID-LAKHDAR Riyane}
\title{Introduction to Cryptology and Coding\\ Lab 2: AES, finite field}
\maketitle
\tableofcontents
\newpage


\section{Introduction}
This document resumes and explains the program that we have implemented to cipher and decipher message using the advanced encryption standard (AES).   This program (attached to this document) may be run by simply running the two following commands:
\begin{itemize}
	\item javac AES.java
    \item java AES
\end{itemize}
Then each question of the topic will be printed separately with an explanation.


\section{Multiplication of a triple by a matrix}
For this task, we have implemented the in the class \textbf{\underline{	"Tools-F-2-power-3"}} the function\\
	\textbf{\underline{	multiplyMatrixByPoly(int[] matrix, int poly, int addPoly, int matrixXSize)
}}
Where each line of the matrix is represented by an entry of the input array, and where the vector (the triple) is simply an integer.\\
All the computations are done modulo 2.  Thus, we don't need to compute any multiplication or addition.   The algorithm will simply compute logical and between each line of the matrix and the vector.   The corresponding line in the vector will simply 1 or 0 depending on whether the number of "1" in the result is even or odd.



\section{SubBytes on a triple}
To implement the function subBytes, we first need to be able to inverse an element of $\mathbb{F}_{2^3}$.\\
To do so, we can notice that the elements of $\mathbb{F}_{2^3}$ form a finite group with respect to multiplication: $a^{p^{n} - 1} = 1$ (where $p = 2$ and $n = 3$).   Thus $a^{-1} = a^{6} $\\
The inverse of an element of $\mathbb{F}_{2^3}$ is done by the function \textbf{\underline{inverse}} in the class \textbf{\underline{Tools-F-2-power-3}}.   This function will simply compute $a^6$ for en entry a, and return the rest of its division by 11 ($a^6$ and 11 are polynomials.  The modulo function of two polynomials is implemented in the same class).\\

Then the subByte function of an input a is given by the formula $A * y + x$ where y is the inverse of a in $\mathbb{F}_{2^3}$, and A and c are the two given matrix and vectors (see function \textbf{\underline{AES.subByte}}).

The full S-box for the input in $\mathbb{F}_{2^3}$ is: ${5, 2, 6, 1, 0, 0, 3, 2}$

\section{Key schedule}
The key schedule function has been implemented in \textbf{\underline{AES.keySchedule}}.   The scheduled key computed by our method is

\begin{equation}
\begin{aligned}
	\mbox{Key}			&= 0 1 1\mbox{   |   }1 0 0\mbox{   |   }1 0 0\mbox{   |   }0 0 1\\
    \mbox{Key round 0}	&= 0 1 1\mbox{   |   }1 0 0\mbox{   |   }1 0 0\mbox{   |   }0 0 1\\
    \mbox{Key round 1}	&= 0 0 1\mbox{   |   }1 0 1\mbox{   |   }0 0 1\mbox{   |   }0 0 0\\
\end{aligned}
\end{equation}





\section{Ciphering}
We have implemented the different steps of the AES algorithm: ShiftRow, MixColumns and AddRoundKey.\\
Thanks to this functions (and to them inverse), we have implemented the cipher functions (and the decipher) in the java function class AES.cipher.\\
The result is:
\begin{equation}
\begin{aligned}
	\mbox{Original message} &= 001 \mbox{  |  } 001 \mbox{  |  } 110 \mbox{  |  } 011\\ 
	\mbox{Ciphered message} &= 010 \mbox{  |  } 001 \mbox{  |  } 110 \mbox{  |  } 110\\ 
\end{aligned}
\end{equation}


\section{Improve the efficiency of the AES}
We can notice that each step of the AES algorithm is deterministic.   We can also notice (by printing the successive keys and ciphered messages) that many combinations of key and messages are repeated.\\
Thus to make our algorithm more efficient, we have stored for each possible input message the result of the successive operations: subByte, shiftRow, and mixColumn.\\
To do so, we have needed for each input message ($2^{12}$ different matrix) one only matrix of the same size.   Thus the memory needed is $2^{12}$ times the size of a message.  Which in our implementation represents a total memory of 65536 bytes.\\

We have on purpose ignored the addRoundKey operation because this operation depends on the key at each round.  Hence, it would make the memory needed sky rocket.\\

The three operations listed bellow had a complexity of $O(n^2)$, where n is the size of the message.   Thus by replacing each of them by a simple lookup in a table, the complexity of the algorithm would become constant (assuming that the xor operation needed for the addRoundKey is done in constant time). 

\end{document}